\begin{frame}{Scaling up to larger systems}
    \begin{figure}
      \centering
      \includegraphics[width=0.5\linewidth]{ef_schematic.pdf}
      \caption{
        Entanglement Forging (EF) exploits the structural properties of the
        simulated system, to decompose a large quantum circuit into smaller
        sub-circuits and knitting their results to estimate the output of the
        larger quantum circuit. In this way, this archetype address the
        limitations of near-future quantum computers (which permit only small,
        and often approximate, simulations over a few qubits with short
        coherence times) and thereof extends the capabilities.
        % Large systems can be broken down into smaller subsystems
        % which can be fit in smaller QCPUS at the cost of classical
        % post-processing, which recovers some of the lost quantum correlations
        % and combines results from the smaller subsystems to approximate to estimate
      }
    \end{figure}
\end{frame}


\begin{frame}{EF results (Noisy simulations)}
    \begin{figure}
      \centering
      \includegraphics[width=0.5\textwidth]{ibm_kyoto_noisy_sim_ef_cobyla_23_pes.pdf}
      \caption{
        Potential energy surface for symmetrically stretched water molecule. In
        the upper plots, EF + VQE results for a noisy simulation of ibm kyoto appear
        alongside curves indicating the classically computed Hartree Fock (HF),
        Complet Active Space Self Consistent Field (CASSCF) for the active
        space (2e,3o) and full configuration interaction (FCI) values using the STO-6G basis.
      }
    \end{figure}
\end{frame}
