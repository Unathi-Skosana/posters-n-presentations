\begin{frame}{Computing Potential Energy Surfaces}
  \begin{columns}
    \begin{column}{0.5\linewidth}
      \begin{itemize}
        \setlength\itemsep{0.1em}
        \item Minima correspond to {\color{red}stable geometry}.
        \item Saddle points correspond to {\color{red}transition states} in chemical reactions.
        \item Identification of energetically favourable reaction pathways.
      \end{itemize}
    \end{column}
    \begin{column}{0.5\linewidth}
      \begin{figure}
        \centering
        \includegraphics[width=1\linewidth]{h2o_symmetric_pes.pdf}
        \caption{
          Schematic of a potential energy surface of a symmetrically stretched water molecule.
          $R$ is varied from smaller $\sim 0.5 \AA$ to $\sim 2.5 \AA$. The bond angle $\alpha$ between
          the hydrogen atoms is fixed at $\alpha = 104.478 ^{\circ}$.
        }
      \end{figure}
    \end{column}
  \end{columns}
\end{frame}


\begin{frame}{Active spaces}
    \begin{figure}
      \centering
      \includegraphics[width=1.0\textwidth]{active_space_diagram.pdf}
      \caption{
        A complete active space splits molecular orbitals into three non-overlapping groups:
        {\color{red}inactive}(or core) orbitals which always be fully occupied (wih two electrons).
        {\color{red}active} orbitals, which can be partially occupied.
        {\color{red}virtual}(or external) orbitals; these must always be empty.
        The notation CAS(n,m) is often used to denote a active space,
        where n is the number of active electrons and m is the number of active orbitals
      }
    \end{figure}

\end{frame}


\begin{frame}{Computing Potential Energy Surfaces}
    \begin{figure}
      \centering
      \includegraphics[width=0.8\linewidth]{h2o_symmetric_benchmark_energies_rohf_sto3g.pdf}
      \caption{
        Potential energy surface for symmetrically stretched water molecule
        at different levels of theory: Full configuration (FCI), Hartree Fock (HF),
        Complete Active Space Self Consistent Field (CASSCF) using minimal STO-3G.
      }
    \end{figure}
\end{frame}



\begin{frame}{Computing Potential Energy Surfaces}
    \begin{figure}
      \centering
      \includegraphics[width=0.8\linewidth]{h2o_symmetric_benchmark_energies_rohf_sto6g.pdf}
      \caption{
        Potential energy surface for symmetrically stretched water molecule
        at different levels of theory: Full configuration (FCI), Hartree Fock (HF),
        Complete Active Space Self Consistent Field (CASSCF) using minimal STO-6G.
      }
    \end{figure}
\end{frame}



\begin{frame}{IBM Quantum Experiments}
  \begin{figure}
    \centering
    \includegraphics[width=0.8\linewidth]{127-qubit-qcpu.pdf}
    \caption{Qubit layout of the 127-qubit ibm\_kyoto (Eagle r3) quantum processor available over the cloud through IBM Quantum API.}
  \end{figure}
\end{frame}

\begin{frame}{IBM Quantum Experiments}
  \begin{figure}
    \centering
    \includegraphics[width=0.7\linewidth]{before_now_circuit_execution.pdf}
    \caption{
      Runtime reduces the waiting time associated with running quantum
      algorithms on real hardware devices. Classical parts of a hybrid are
      executed on an IBM Server near the quantum hardware, reducing latency.
    }
    % executing the classical parts of a hybrid
    % quantum-classical algorithm on an IBM server closely located to the
    % actual device, thereby minimizing communication overhead.
  \end{figure}
\end{frame}

\begin{frame}{Error mitigation: Readout mitigation}
  Assumption: Measurement errors satisfy

  \begin{align*}
    p_\text{noisy} &=  \Lambda p_\text{ideal},
  \end{align*}

  where $p_\text{ideal}$ and $p_\text{noisy}$ are vector probabilities arranged
  into a matrix returned from computation without and with measurement errors
  respectively, and $\Lambda$ is an invertible matrix.

  \vspace{2ex}

  Execute appropriate $2^n$ circuits noisy quantum hardware to measure
  $p_\text{noisy}$. Estimate $\Lambda^{-1}$ as $\tilde{\Lambda}^{-1}$ with {\color{red}Quantum Detector
  Tomography}. Then

  \begin{align*}
    p_\text{mitigated} \approx p_\text{ideal} &= \tilde{\Lambda}^{-1}p_\text{noisy}.
  \end{align*}


  % If
  % $M$ is invertible then, then $C_\text{noisy}$  can transformed into
  % % $C_\text{ideal}$ by finding $M^{-1}$
  %
  % \begin{align}
  %   C_\text{ideal} &=  M^{-1}C_\text{noisy}.
  % \end{align}
  % In reducing the effect of noise due to final measurement errors, Qiskit
  % recommend a measurement error mitigation approach. The approach starts off by
  % creating circuits that each perform measurement of the $2^n$ basis states. With
  % the results, a matrix $C_\text{noisy}$ called the calibration matrix is
  % constructed. The approach assumes that there is a matrix $M$ such that
\end{frame}


\begin{frame}{Error mitigation: Zero Noise Extrapolation}
   \begin{figure}
    \centering
    \includegraphics[width=1\linewidth]{noise_amplification.pdf}
     \caption{Global noise amplification: Amplifies noise digitally by "folding" each gate to a desired noise level.}
  \end{figure}
\end{frame}


\begin{frame}{Error mitigation: Zero Noise Extrapolation}
  \begin{figure}
    \centering
    \includegraphics[width=0.7\linewidth]{zne_schematic.pdf}
    \caption{Extrapolatation to the noiseless limit: Fit a function $f$ to the measured noise-scaled expectation values and extrapolate to limit of no noise.}
  \end{figure}
\end{frame}


\begin{frame}{Classical optimization algorithms}
  \begin{itemize}
    \setlength\itemsep{0.1em}
    \item {\color{red}COBYLA} uses linear approximations of the target and constraint function to optimize a simplex within a trust region of the parameter space.
        \begin{itemize}
          \item Evaluates the cost function without use of gradients.
          \item Susceptible to get stuck in local minima.
          \item Relatively simple to implement.
        \end{itemize}
    \item {\color{red}SPSA} employs a stochastic perturbation vector to compute an approximate gradient of the cost function simultaneously.
        \begin{itemize}
          \item Works well for noisy functions.
          \item An SPSA step only requires two cost function evaluations; independent of the number of parameters for the cost function.
          %% (Making it applicable to problems with many dimensions.)
          \item Performance highly-dependent on the hyperparameters.
          %% (especially step and perturbation sizes.)
          \item Can be easily parallelized.
        \end{itemize}
  \end{itemize}
\end{frame}
