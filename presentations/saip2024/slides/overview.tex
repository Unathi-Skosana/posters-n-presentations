%     \item To understand the practical aspects of modeling and simulating small-scale molecular systems with quantum computing methods on near-term quantum computing hardware.
%     \item In the near-term, one of the most promising and viable quantum computing methods for doing thus are variational quantum algorithms (VQAs) in conjunction with error mitigation techniques.
%
\begin{frame}{Original aims}
    To understand how to perform quantum simulations efficiently on current small-scale {\color{red}Noisy Intermediate Scale Quantum (NISQ)} processors and how these simulations
    can be scaled up for forthcoming large-scale processors in the near future.
\end{frame}

\begin{frame}{Original objectives}
    The objectives for the first part of the project involve using new {\color{red} Quantum Phase Estimation (QPE)} techniques and comparing their performance
    with the {\color{red}Variational Quantum Eigensolver (VQE)} method on IBM quantum processors for a selection of quantum chemistry systems:

    \vspace{2ex}

    \begin{enumerate}
      \setlength\itemsep{0.1em}
      \item[O1] Basic system – H2
      \item[O2] Medium system – H2O
      % \item[O3] Hard system – Chlorophyll units in LHII complex or a system of a similar
    \end{enumerate}

    \vspace{2ex}

    We soon learned that QPE techniques are quite resource-intensive, even for a basic molecular system such as H2; \textbf{i.e.}
    decimal point accuracy of calculated quantities with QPE is in direct proportion to the number of phyisical qubits needed. Henceforth,
    we only focused on VQE techniques for objectives O1 and O2.
\end{frame}


\begin{frame}{O1}
  \begin{itemize}
    \setlength\itemsep{0.1em}
    \item H2 system is simplest instance we considered. Electronic hamiltonian of this system is mapped to a 4-qubit Hamiltonian in the minimal basis STO-3G.
    \item We experimentally computed the electronic energies of the H2 system at varying interatomic distances (points sampled between {\color{teal}0.4 to 3.5}).
    \item We successfully carried out with VQE experiments on a physical device (\textbf{ibmq\_guadalupe}).
    \item Obtaining experimental results that closely matche the expected results computed classically via PySCF.
    \item The experiments are carried over the cloud through the Qiskit SDK which includes Qiskit Nature for Quantum Chemistry
    \item I have since then had contributed two significant occasions now :D!
  \end{itemize}
\end{frame}

\begin{frame}{O1}
  \begin{itemize}
    \setlength\itemsep{0.1em}
    \item The first instance of the experiment was done without any further simplifications; requiring a total of 4 qubits.
    \item The second instance of the experiment, the qubit Hamiltonian was simplified to 2 qubits by exploiting present symmetries.
  \end{itemize}
\end{frame}

\begin{frame}{O1: Lessons}
  \begin{itemize}
    \setlength\itemsep{0.1em}
    \item These set of experiments served a pedagogic purpose without being encumbered with the complexities of large molecular systems,
          we encountered later in the project.
    \item We achieved results very close to {\color{red}chemical accuracy ($\Delta E < 1.6 \>\text{mHa}$)} without any error mitigation
          techniques.
    \item Problem reduction techniques such as exploiting symmetry to reduce problem size
          are useful in getting a handle at a problem on near-term hardware, and will be reoccuring theme in various guises.
  \end{itemize}
\end{frame}


\begin{frame}{O2}
  \begin{itemize}
    \setlength\itemsep{0.1em}
    \item In the minimal basis STO-3G, electronic Hamiltonian of H2O is mapped to a 14-qubit Hamiltonian.
    \item We wished to compute the electronic energies of the H2O system along the reaction coordinates
          that symmetrically stretch the hydrogen atoms around oxygen atom while keeping the angle between them fixed.
    \item However, for IBM Quantum devices available on the cloud at the time, maintaining coherence over this many qubits for this kind of problem was not trivial.
    \item Instead of considering all the orbitals in molecular system, we consider only a subset of orbitals.
  \end{itemize}
\end{frame}


\begin{frame}{O2: Active Spaces}
  \begin{itemize}
    \setlength\itemsep{0.1em}
    \item The set of orbitals and the number of electrons in them is called an {\color{red} Active Space (AS)}.
          An active space of 2 electrons and 3 orbitals is denoted (2e,3o).
    \item The energy contributions from the AS are computed on a quantum computer using VQE,
          the contributions from orbitals outside the AS are computed classically.
    \item The two contributions are added together to obtain the approximate total electronic energy.
    \item If the subset is chosen carefully (i.e automated with {\color{blue}AutoCAS, AVAS}) energies calculated in this manner can be in good qualitative
          agreement with energies calcuated from a full set of orbitals.
  \end{itemize}
\end{frame}


\begin{frame}{O2: Active Spaces}
  \begin{itemize}
    \setlength\itemsep{0.1em}
    \item Using PySCF, we classically computed energies obtained along the aforemetioned reaction coordinates
          using a subset of orbitals from size 2 to 10.
    \item Found that we approach {\color{red}chemical accuracy ($\Delta E < 1.6 \>\text{mHa}$)} for the minimal bases STO-3G/STO-6G.
    \item Active spaces are chosen around the {\color{red}Fermi level}, unless mentioned otherwise.
  \end{itemize}
\end{frame}


\begin{frame}{O2: Lessons}
  \begin{itemize}
    \setlength\itemsep{0.1em}
    \item Starting from two of the smallest active spaces; (2e,3o) and (4e,4o); we conducted VQE experiments on various IBM Quantum devices obtained a mixed bag of results.

    \item The choice of ansatz matters significantly, as it determines the possible quantum states can be accessed during VQE optimization.

    \item Through a heuristic process of trying many different ansatzes,
      I have found that heuristic {\color{red}two-local} ansatz with Hop gates $h(\varphi)$ as entangling blocks tends to the best:
      \begin{align*}
        h(\varphi) =
          \begin{bmatrix}
            1 & 0               &  0                & 0 \\
            0 & \cos{(\varphi)} & -\sin{(\varphi)}  & 0 \\
            0 & \sin{(\varphi)} &  \cos{(\varphi)}  & 0 \\
            0 & 0               &  0                & -1
          \end{bmatrix}.
      \end{align*}

    \item A product of hop gates is universal with respect to real-valued wavefunctions of fixed-particle number.
  \end{itemize}
\end{frame}

\begin{frame}{O2: Lessons}
  \begin{itemize}
    \setlength\itemsep{0.1em}
    \item Hence, such an ansatz is a hybrid between being hardware efficient and chemically inspired ansatz.  We refer to such an ansatz as a heuristic ansatz.

    \item Another note-worthy heuristic ansatz is excitation preserving made up of entangling blocks of the form
      \begin{align*}
        e(\theta, \phi) =
          \begin{bmatrix}
            1 & 0                  &  0                  & 0 \\
            0 & \cos{(\theta/2)}   & -i\sin{(\theta/2)}  & 0 \\
            0 & -i\sin{(\theta/2)} &  \cos{(\theta/2)}   & 0 \\
            0 & 0                  &  0                  & e^{-i\phi}
          \end{bmatrix}.
      \end{align*}

    \item The choice of optimization algorithm matters. In my experimentation, I have found that gradient-free optimizers
          such as Constrained Optimization By Linear Approximation (COBYLA) tend work well.

  \end{itemize}
\end{frame}

\begin{frame}{O2: Lessons}
  \begin{itemize}
    \setlength\itemsep{0.1em}
    \item COBYLA tends to be stable and converge with fewer iterations than gradient-based optimizer
          such as Simultaneous Perturbative Stochastic Approximation (SPSA).

    \item When using two-local ansatz with $\text{R}_{Y}(\theta)$ and CZ as rotation and entanglement blocks, respectively.
          Around 1.5 $\AA$ (consistent among multiple plots), the energies spike up and jump roughly by 0.5 Hartrees.

    \item The same bumps show up in literature. Explained as being due to jumps between electronic states of different symmetries ($\expval*{\hat{S}}, \expval*{\hat{N}_e}, \expval*{\hat{S}_z}$) within the Fock space.

    \item Bumps can be mitigated performing penalty constrained optimization.

  \end{itemize}
\end{frame}

\begin{frame}{O2: Lessons}
  \begin{itemize}
    \setlength\itemsep{0.1em}
    \item Circuit knitting techniques such as {\color{red}Entanglment Forging (EF)} can also reduce the size of the problem
          to be treated on a quantum computer by shifting some of the commputation to classical post-processing.

    \item For instance with entanglment forging, a suitable problem of requiring 6 qubits i.e can be reduced to a problem on 3 qubits bar classical overhead.

    \item We conducted VQE experiments on various IBM Quantum devices obtained a mixed bag of results.

  \end{itemize}
\end{frame}

\begin{frame}{O2: Lessons}
  \begin{itemize}
    \setlength\itemsep{0.1em}
    \item Circuit knitting techniques such as {\color{red}Entanglment Forging (EF)} can also reduce the size of the problem
          to be treated on a quantum computer by shifting some of the commputation to classical post-processing.

    \item For instance with entanglment forging, a suitable problem of requiring 6 qubits i.e can be reduced to a problem on 3 qubits bar classical overhead.

    \item Similarly, we conducted VQE+EF experiments on a IBM Quantum devices over cloud. However, we obtained an incomplete set of results that are not conclusive.

    \item This was due to variety of reasons ...

  \end{itemize}
\end{frame}

\begin{frame}{O2: Blockers}
  \begin{itemize}
    \setlength\itemsep{0.1em}
    \item The turnaround time between generating an idea, testing it on a physical processor,
          become notably longer due to popularity of the service.

    \item While our current quantum computational workflows have grown in complexity and computational demands
      i.e compared to Shor's algorithm workflow from our previous paper

    \item The introduction of the Qiskit Runtime resulted in significant changes and breaking updates. Slowly downing progress

    \item IBM Quantum subscription was suspended for a lengthy period of time.
  \end{itemize}
\end{frame}

\begin{frame}{O2: New ideass}
  \begin{itemize}
    \setlength\itemsep{0.1em}
    \item Incorporate hyperparameter tuning step at the begining of the VQE workflow.

    \item {\color{red}Hyperparameter tuning} is necessary for inherently-noisy VQAs.

    \item Variational landscape that one finds in a VQA is highly problem and device-dependent.

    \item Ideally, hyperparameter tuning runs would performed on a real physical device too.
          However, this infeasible for us as our time on IBM Quantum is very limited.

    \item Instead, we checkpoint the noise model of a particular device and classically hyperparameter tuning experiments.

    \item Once we have found the best hyperparameters to initialize the chosen optimization algorithm for an experiment on a physical device.
  \end{itemize}
\end{frame}


\begin{frame}{O2: New ideas}
  \begin{itemize}
    \setlength\itemsep{0.1em}
    \item During the time when IBM Quantum subscription was suspended. I rewrote the code we had been using thus far
          to incorporate hyperparameter tuning. Hyperparameter tuning is done with {\color{blue}Optuna}.

    \item The code also standardizes the data storage of experimental data i.e stored inside local {\color{blue}SQLite} 
          databases.

    \item The progress of experiments can be tracked remotely through {\color{blue}WandB},
          which also stores the experimental data locally.

    \item The new codebase with all these additions is called {\color{red}VQUNE} and will open-source it at some point.
  \end{itemize}
\end{frame}

\begin{frame}{O2: New experiments}
  \begin{itemize}
    \setlength\itemsep{0.1em}

    \item We designed a new set of experiments which incorporate hyperparameter tuning: which use COBYLA and SPSA as optimization algorithms with and without ZNE .

    \item Additionally, our plight forced to considered a new and much reduced set of reaction coordinates
          (hydrogen atoms are symmetrically stretched around oxygen atom while keeping the angle between them fixed)

    \item During a period of roughly three weeks, we were able to perform these some of experiments.

    \item For experiments using the COBYLA optimization algorithm, the new set of results show significant improvements when hyperparameter tuning is done beforehand.

    \item For experiments using the SPSA optimization algorithm, the new set of results are incomplete and still being collected.
  \end{itemize}
\end{frame}
