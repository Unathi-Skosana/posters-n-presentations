\begin{frame}{Summary}
   \begin{itemize}
    \setlength\itemsep{0.1em}
    \item In our real quantum device experiments, the combination of a hyperparameter-tuned {\color{red}COBYLA} optimizer with zero noise extrapolation
          seems to produce results are comparatively better than those from a hyperparameter-tuned COBYLA optimizer without zero noise extrapolation.

    \item Hence, even with hyperparameter tuning raw experimental results [without zero noise extrapolation] can be below par.

    \item Error mitigation (measurement error mitigation + ZNE) seems to remove undesirable features in PES, such as the observed kinks/dumps on jarkata near disassociation

  \end{itemize}
\end{frame}

\begin{frame}{Summary}
   \begin{itemize}
    \setlength\itemsep{0.1em}
    \item For classically intracble molecular systems, hyperparameter tuning can be done online as part of the VQE pipeline/workflow
          i.e performed on the on quantum device of interest.

    \item For classically tractable molecular systems, one is has the option of doing either online and offline hyperparameter
          tuning, in our studies we found that offline hyperparameter tuning can significantly improve the performance
          of the {\color{red}COBYLA} optimizer (and other optimizers such as SPSA) for error-mitigated and non-error-mitigated landscapes
          on a real quantum device. We suspect that performance would be improved further if done online.
   \end{itemize}
\end{frame}

\begin{frame}{Summary}
   \begin{itemize}
    \setlength\itemsep{0.1em}
    \item For a molecular system where both offline and online hyperparameter tuning are feasible,
          the choice is mostly influenced by practical considerations, i.e. amount quantum
          compute time available. In our case, we had limited quantum compute time, hence
          offline hyperparameter tuning.

    \item It may well be that there are other techniques that one can use on a smaller
          truncated molecular system with similar chemical/molecular
          features that allow efficient tuning on classical and quantum computers.


   % Note: Even with hyperparameter tuning, raw exp. results can be bad. Seems like the combination of tuning + ZNE works better
   % just that one can simulate a few runs of the system classically to get the hyperparameters roughly,
   % or use up some of the runs on a quantum computer before doing the full
   % optimisation routine.
   \end{itemize}
\end{frame}
