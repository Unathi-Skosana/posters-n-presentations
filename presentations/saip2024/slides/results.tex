\begin{frame}{Hyperparameter tuning}
    Variational landscapes for VQAs are problem-specific and device-dependent.
    Performance of the off the shell optimizers can vary drastically if not
    tuned to the problem at hand.

    \begin{figure}
      \centering
      \includegraphics[width=0.75\linewidth]{energy_landscapes.pdf}
      \caption{
        Energy landscapes showing the effects of certain types of noise models.
        Adapted from \href{https://quantum-journal.org/papers/q-2024-03-14-1287/pdf}{here}
      }
    \end{figure}

\end{frame}

\begin{frame}{Hyperparameter tuning}

    Hyperparameter tuning can be {\color{red}offline}; that is, before the actual
    experiments to get the best hyperparameters using a noise model of the device
    of interest. This is working under the assumption that the molecualr system of interest
    is small enough (requiring a moderate number of qubits) to be classically simulated.

    \vspace{2ex}

    It can also be done {\color{red}online}, where the tuning happens on the actual device of
    interest. i.e. the budget of the original experiments is split between
    tuning and training runs, where a small fraction of the budget is spent
    on the expanded tuning runs prior to training runs.

    % online means that a linear overhead in time for testing is
    % assumed - otherwise the computational time becomes inefficient.

    % Might get you roughly close to the optimal value for the
    % the hyperparameters  within the constrained budget, but not
    % all the way.
\end{frame}

\begin{frame}{VQE with Hyperparameter tuning}
    \begin{figure}
      \centering
      \includegraphics[height=0.6\linewidth]{vqe_flowchart_with_optuna.pdf}
      \caption{
        Hyperparameter tuning can be seamlessly incorporated into the VQE pipeline.
        In our case, we use the automatic hyperparameter tuning framework optuna.
      }
    \end{figure}
\end{frame}


\begin{frame}{VQE Results (Without hyperparameter tuning)}
    \begin{figure}
      \centering
      \includegraphics[width=0.7\linewidth]{ibm_jarkata_cobyla_23_pes.pdf}
      \caption{
        Potential energy surface for symmetrically stretched water molecule. In
        the upper plots, VQE results without ZNE on ibm jarkata appear
        alongside curves indicating the classically computed Hartree Fock (HF),
        Complet Active Space Self Consistent Field (CASSCF) for the active
        space (2e,3o) and full configuration interaction (FCI) values using the STO-3G basis.
        Lower plots indicate absolute errors, $|E − E_\text{CASSCF}|$.
      }
      % at different levels of theory computed on ibm\_kyoto  Full configuration (FCI), Hartree Fock (HF),
      % Complete Active Space Self Consistent Field (CASSCF) using minimal STO-6G.
    \end{figure}
\end{frame}

\begin{frame}{VQE Results (With hyperparameter tuning)}
    \begin{figure}
      \centering
      \includegraphics[width=0.7\linewidth]{ibm_kyoto_cobyla_23_pes.pdf}
      \caption{
        Potential energy surface for symmetrically stretched water molecule. In
        the upper plots, VQE results with and without ZNE on ibm kyoto appear
        alongside curves indicating the classically computed Hartree Fock (HF),
        Complet Active Space Self Consistent Field (CASSCF) for the active
        space (2e,3o) and full configuration interaction (FCI) values using the STO-6G basis.
        Lower plots indicate absolute errors, $|E − E_\text{CASSCF}|$.
      }
    \end{figure}
\end{frame}


\begin{frame}{VQE Results (With hyperparameter tuning)}
  In terms of hardware, ibm kyoto has a much better spec than ibm jarkata.
  Despite this, hyperparameter tuning on its own does not result in great
  improvement in the results, if at all across the two devices.

  \begin{figure}
      \centering
      \includegraphics[width=.55\linewidth]{../gfx/ibm_kyoto_cobyla_23_optimization_history_nsgaii_median_hyperparameter_offline_d_1.0.pdf}
      \caption{Energy vs trial number during COBYLA hyperparameter tuning at the reaction coordinate $R = 1.0$. This is for the active space (2e,3o) using the STO-6G basis set.}
  \end{figure}
\end{frame}


\begin{frame}{VQE Results (With hyperparameter tuning)}
  \begin{figure}
      \centering
      \includegraphics[width=.6\linewidth]{../gfx/ibm_kyoto_cobyla_23_intermediate_values_nsgaii_median_hyperparameter_offline_d_1.0.pdf}
      \caption{Energy vs trial number during COBYLA hyperparameter tuning at the reaction coordinate $R = 1.0$. This is for the active space (2e,3o) using the STO-6G basis set.}
  \end{figure}
\end{frame}

\begin{frame}{VQE Results (With hyperparameter tuning)}
  \begin{figure}
      \centering
      \includegraphics[width=.6\linewidth]{../gfx/ibm_kyoto_spsa_23_optimization_history_nsgaii_median_hyperparameter_offline_d_1.0.pdf}
      \caption{Energy vs trial number during SPSA hyperparameter tuning at the reaction coordinate $R = 1.0$. This is for the active space (2e,3o) using the STO-6G basis set.}
  \end{figure}
\end{frame}


\begin{frame}{VQE Results (With hyperparameter tuning)}
  \begin{figure}
      \centering
      \includegraphics[width=.6\linewidth]{../gfx/ibm_kyoto_spsa_23_intermediate_values_nsgaii_median_hyperparameter_offline_d_1.0.pdf}
      \caption{Energy vs trial number during SPSA hyperparameter tuning at the reaction coordinate $R = 1.0$. This is for the active space (2e,3o) using the STO-6G basis set.}
  \end{figure}
\end{frame}
